\documentclass{beamer}
\usepackage{pgfpages}
\usepackage[backend=bibtex]{biblatex}
\usepackage{multicol}
\usepackage{multimedia}
\usepackage[absolute,overlay]{textpos}
\usepackage{parskip}
\usepackage{hyperref}
\usepackage{lmodern}
\hypersetup{colorlinks=true, urlcolor=blue}
\setlength{\parskip}{\smallskipamount} 
%\usepackage[texcoord,grid,gridunit=mm,gridcolor=red!10,subgridcolor=green!10]{eso-pic} %DELETE when done with grid
\setbeameroption{hide notes} % Only slides
%\setbeameroption{show only notes} % Only notes
%\setbeameroption{show notes on second screen=right} % Both
%\bibliography{../../papers/references.bib}
\setbeamerfont{footnote}{size=\tiny}
%\AtEveryCitekey{\clearfield{title}}

%
% Choose how your presentation looks.
%
% For more themes, color themes and font themes, see:
% http://deic.uab.es/~iblanes/beamer_gallery/index_by_theme.html
%
\mode<presentation>
{
  \usetheme{Warsaw}      % or try Darmstadt, Madrid, Warsaw, ...
  \usecolortheme{wolverine} % or try albatross, beaver, crane, ...
  \usefonttheme{default}  % or try serif, structurebold, ...
  \setbeamertemplate{navigation symbols}{}
  \setbeamertemplate{caption}[numbered]
} 

\usepackage[english]{babel}
%\usepackage[utf8x]{inputenc} %Doesn't play well with biblatex
\usepackage{amssymb}
\usepackage{bm}
\usepackage{color}
\usepackage{graphicx}

\newcommand{\red}[1]{{\color{red}{#1}}}

\title[{\color{blue}{Day 2}}]{{Day 2: RGB and Synesthesia}}
\author{Python Activity}
%\institute{Universidad Aut\'onoma de Baja California}
\date{}

\begin{document}

%\setbeamertemplate{frametitle}[default][center]
\begin{frame}
   \titlepage
\end{frame}

% Uncomment these lines for an automatically generated outline.
%\begin{frame}{Outline}
%  \tableofcontents
%\end{frame}

% Commands to include a figure:
%\begin{figure}
%\includegraphics[width=\textwidth]{your-figure's-file-name}
%\caption{\label{fig:your-figure}Caption goes here.}
%\end{figure}

\begin{frame}{Hyper-spectral images and Synesthesia}
   \begin{itemize}
      \item Learning Objectives
      \begin{itemize}
         \item Sensorial
         \begin{itemize}
            \item Be able to identify the hyper spectral components of a pixel based on its ``sound".
         \end{itemize}
         \item Technical
         \begin{itemize}
            \item Usage of python dictionaries.
            \item Manipulation of Hyperspectral information.
            \item Explore different ways of data compression beyond a single channel.
         \end{itemize}
      \end{itemize}
   \end{itemize}
\end{frame}

\begin{frame}{Activities}
   \begin{enumerate}
      \item Hyper-spectral images:
      \begin{itemize}
         \item Complete the notebook called Explore\_Hyperspectral\_Image.ipynb.
         \begin{itemize}
            \item If you don't have all the needed packages try \\ \texttt{pip install <package\_name>}
         \end{itemize}
         \item Understand script Hyperspectral\_Synesthesia.py
         \begin{itemize}
            \item Explore the script.
            \item Add more wavelengths/sounds in a single channel.
         \end{itemize}
      \end{itemize}
      \item If you have extra time:
      \begin{itemize}
         \item Work though the TIP worksheets
         \begin{itemize}
            \item 1. RGB\_TIP.ipynb
            \item 2. Hyperspectral\_TIP.ipynb
         \end{itemize}
      \end{itemize}
   \end{enumerate}
\end{frame}

\begin{frame}{Install libraries on your own computer}
If you want to try the scripts and notebooks in your own computer you can follow these instructions in Mac:
\begin{itemize}
   \item Get python and some modules
   \begin{itemize}
      \item Go to \href{https://www.continuum.io/downloads\#osx}{https://www.continuum.io/downloads\#osx}
      \item Download graphical installer python 2.7 version.
      \item Open a terminal and type
      \begin{itemize}
         \item \texttt{pip install pygame}
         \item \texttt{pip install imageio}
         \item \texttt{pip install pydub}
         \item \texttt{pip install jupyter}
      \end{itemize}
   \end{itemize}
   \item Run jupyter notebook:
   \begin{itemize}
      \item In a terminal type \texttt{jupyter notebook}
      \item Find and select the desired .ipynb
   \end{itemize}
\end{itemize}
\end{frame}

\begin{frame}{Some notes on iPython notebooks}
Every time you open a notebook the kernel should be Python 2, or Python or Python[default] (it would depend on how many python installation you have and your jupyter config file). If you want to select a kernel go to kernel $\rightarrow$ change kernel $\rightarrow$ and select the correct python version.

If are not familiar with IPython notebooks a couple of minutes of the following video could be useful (after the minute 4:00, before 5:30):
\begin{center}
   \href{https://youtu.be/irJVUeYlJgU?t=4m}{https://youtu.be/irJVUeYlJgU?t=4m}
\end{center}
It is about the basics: how to run, stop, change kernels, and have a tour of the interface.
\end{frame}

\end{document}
